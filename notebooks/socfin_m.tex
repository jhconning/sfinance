\documentclass[11pt]{article}

    \usepackage[breakable]{tcolorbox}
    \usepackage{parskip} % Stop auto-indenting (to mimic markdown behaviour)
    

    % Basic figure setup, for now with no caption control since it's done
    % automatically by Pandoc (which extracts ![](path) syntax from Markdown).
    \usepackage{graphicx}
    % Keep aspect ratio if custom image width or height is specified
    \setkeys{Gin}{keepaspectratio}
    % Maintain compatibility with old templates. Remove in nbconvert 6.0
    \let\Oldincludegraphics\includegraphics
    % Ensure that by default, figures have no caption (until we provide a
    % proper Figure object with a Caption API and a way to capture that
    % in the conversion process - todo).
    \usepackage{caption}
    \DeclareCaptionFormat{nocaption}{}
    \captionsetup{format=nocaption,aboveskip=0pt,belowskip=0pt}

    \usepackage{float}
    \floatplacement{figure}{H} % forces figures to be placed at the correct location
    \usepackage{xcolor} % Allow colors to be defined
    \usepackage{enumerate} % Needed for markdown enumerations to work
    \usepackage{geometry} % Used to adjust the document margins
    \usepackage{amsmath} % Equations
    \usepackage{amssymb} % Equations
    \usepackage{textcomp} % defines textquotesingle
    % Hack from http://tex.stackexchange.com/a/47451/13684:
    \AtBeginDocument{%
        \def\PYZsq{\textquotesingle}% Upright quotes in Pygmentized code
    }
    \usepackage{upquote} % Upright quotes for verbatim code
    \usepackage{eurosym} % defines \euro

    \usepackage{iftex}
    \ifPDFTeX
        \usepackage[T1]{fontenc}
        \IfFileExists{alphabeta.sty}{
              \usepackage{alphabeta}
          }{
              \usepackage[mathletters]{ucs}
              \usepackage[utf8x]{inputenc}
          }
    \else
        \usepackage{fontspec}
        \usepackage{unicode-math}
    \fi

    \usepackage{fancyvrb} % verbatim replacement that allows latex
    \usepackage{grffile} % extends the file name processing of package graphics
                         % to support a larger range
    \makeatletter % fix for old versions of grffile with XeLaTeX
    \@ifpackagelater{grffile}{2019/11/01}
    {
      % Do nothing on new versions
    }
    {
      \def\Gread@@xetex#1{%
        \IfFileExists{"\Gin@base".bb}%
        {\Gread@eps{\Gin@base.bb}}%
        {\Gread@@xetex@aux#1}%
      }
    }
    \makeatother
    \usepackage[Export]{adjustbox} % Used to constrain images to a maximum size
    \adjustboxset{max size={0.9\linewidth}{0.9\paperheight}}

    % The hyperref package gives us a pdf with properly built
    % internal navigation ('pdf bookmarks' for the table of contents,
    % internal cross-reference links, web links for URLs, etc.)
    \usepackage{hyperref}
    % The default LaTeX title has an obnoxious amount of whitespace. By default,
    % titling removes some of it. It also provides customization options.
    \usepackage{titling}
    \usepackage{longtable} % longtable support required by pandoc >1.10
    \usepackage{booktabs}  % table support for pandoc > 1.12.2
    \usepackage{array}     % table support for pandoc >= 2.11.3
    \usepackage{calc}      % table minipage width calculation for pandoc >= 2.11.1
    \usepackage[inline]{enumitem} % IRkernel/repr support (it uses the enumerate* environment)
    \usepackage[normalem]{ulem} % ulem is needed to support strikethroughs (\sout)
                                % normalem makes italics be italics, not underlines
    \usepackage{soul}      % strikethrough (\st) support for pandoc >= 3.0.0
    \usepackage{mathrsfs}
    

    
    % Colors for the hyperref package
    \definecolor{urlcolor}{rgb}{0,.145,.698}
    \definecolor{linkcolor}{rgb}{.71,0.21,0.01}
    \definecolor{citecolor}{rgb}{.12,.54,.11}

    % ANSI colors
    \definecolor{ansi-black}{HTML}{3E424D}
    \definecolor{ansi-black-intense}{HTML}{282C36}
    \definecolor{ansi-red}{HTML}{E75C58}
    \definecolor{ansi-red-intense}{HTML}{B22B31}
    \definecolor{ansi-green}{HTML}{00A250}
    \definecolor{ansi-green-intense}{HTML}{007427}
    \definecolor{ansi-yellow}{HTML}{DDB62B}
    \definecolor{ansi-yellow-intense}{HTML}{B27D12}
    \definecolor{ansi-blue}{HTML}{208FFB}
    \definecolor{ansi-blue-intense}{HTML}{0065CA}
    \definecolor{ansi-magenta}{HTML}{D160C4}
    \definecolor{ansi-magenta-intense}{HTML}{A03196}
    \definecolor{ansi-cyan}{HTML}{60C6C8}
    \definecolor{ansi-cyan-intense}{HTML}{258F8F}
    \definecolor{ansi-white}{HTML}{C5C1B4}
    \definecolor{ansi-white-intense}{HTML}{A1A6B2}
    \definecolor{ansi-default-inverse-fg}{HTML}{FFFFFF}
    \definecolor{ansi-default-inverse-bg}{HTML}{000000}

    % common color for the border for error outputs.
    \definecolor{outerrorbackground}{HTML}{FFDFDF}

    % commands and environments needed by pandoc snippets
    % extracted from the output of `pandoc -s`
    \providecommand{\tightlist}{%
      \setlength{\itemsep}{0pt}\setlength{\parskip}{0pt}}
    \DefineVerbatimEnvironment{Highlighting}{Verbatim}{commandchars=\\\{\}}
    % Add ',fontsize=\small' for more characters per line
    \newenvironment{Shaded}{}{}
    \newcommand{\KeywordTok}[1]{\textcolor[rgb]{0.00,0.44,0.13}{\textbf{{#1}}}}
    \newcommand{\DataTypeTok}[1]{\textcolor[rgb]{0.56,0.13,0.00}{{#1}}}
    \newcommand{\DecValTok}[1]{\textcolor[rgb]{0.25,0.63,0.44}{{#1}}}
    \newcommand{\BaseNTok}[1]{\textcolor[rgb]{0.25,0.63,0.44}{{#1}}}
    \newcommand{\FloatTok}[1]{\textcolor[rgb]{0.25,0.63,0.44}{{#1}}}
    \newcommand{\CharTok}[1]{\textcolor[rgb]{0.25,0.44,0.63}{{#1}}}
    \newcommand{\StringTok}[1]{\textcolor[rgb]{0.25,0.44,0.63}{{#1}}}
    \newcommand{\CommentTok}[1]{\textcolor[rgb]{0.38,0.63,0.69}{\textit{{#1}}}}
    \newcommand{\OtherTok}[1]{\textcolor[rgb]{0.00,0.44,0.13}{{#1}}}
    \newcommand{\AlertTok}[1]{\textcolor[rgb]{1.00,0.00,0.00}{\textbf{{#1}}}}
    \newcommand{\FunctionTok}[1]{\textcolor[rgb]{0.02,0.16,0.49}{{#1}}}
    \newcommand{\RegionMarkerTok}[1]{{#1}}
    \newcommand{\ErrorTok}[1]{\textcolor[rgb]{1.00,0.00,0.00}{\textbf{{#1}}}}
    \newcommand{\NormalTok}[1]{{#1}}

    % Additional commands for more recent versions of Pandoc
    \newcommand{\ConstantTok}[1]{\textcolor[rgb]{0.53,0.00,0.00}{{#1}}}
    \newcommand{\SpecialCharTok}[1]{\textcolor[rgb]{0.25,0.44,0.63}{{#1}}}
    \newcommand{\VerbatimStringTok}[1]{\textcolor[rgb]{0.25,0.44,0.63}{{#1}}}
    \newcommand{\SpecialStringTok}[1]{\textcolor[rgb]{0.73,0.40,0.53}{{#1}}}
    \newcommand{\ImportTok}[1]{{#1}}
    \newcommand{\DocumentationTok}[1]{\textcolor[rgb]{0.73,0.13,0.13}{\textit{{#1}}}}
    \newcommand{\AnnotationTok}[1]{\textcolor[rgb]{0.38,0.63,0.69}{\textbf{\textit{{#1}}}}}
    \newcommand{\CommentVarTok}[1]{\textcolor[rgb]{0.38,0.63,0.69}{\textbf{\textit{{#1}}}}}
    \newcommand{\VariableTok}[1]{\textcolor[rgb]{0.10,0.09,0.49}{{#1}}}
    \newcommand{\ControlFlowTok}[1]{\textcolor[rgb]{0.00,0.44,0.13}{\textbf{{#1}}}}
    \newcommand{\OperatorTok}[1]{\textcolor[rgb]{0.40,0.40,0.40}{{#1}}}
    \newcommand{\BuiltInTok}[1]{{#1}}
    \newcommand{\ExtensionTok}[1]{{#1}}
    \newcommand{\PreprocessorTok}[1]{\textcolor[rgb]{0.74,0.48,0.00}{{#1}}}
    \newcommand{\AttributeTok}[1]{\textcolor[rgb]{0.49,0.56,0.16}{{#1}}}
    \newcommand{\InformationTok}[1]{\textcolor[rgb]{0.38,0.63,0.69}{\textbf{\textit{{#1}}}}}
    \newcommand{\WarningTok}[1]{\textcolor[rgb]{0.38,0.63,0.69}{\textbf{\textit{{#1}}}}}


    % Define a nice break command that doesn't care if a line doesn't already
    % exist.
    \def\br{\hspace*{\fill} \\* }
    % Math Jax compatibility definitions
    \def\gt{>}
    \def\lt{<}
    \let\Oldtex\TeX
    \let\Oldlatex\LaTeX
    \renewcommand{\TeX}{\textrm{\Oldtex}}
    \renewcommand{\LaTeX}{\textrm{\Oldlatex}}
    % Document parameters
    % Document title
    \title{socfin\_m}
    
    
    
    
    
    
    
% Pygments definitions
\makeatletter
\def\PY@reset{\let\PY@it=\relax \let\PY@bf=\relax%
    \let\PY@ul=\relax \let\PY@tc=\relax%
    \let\PY@bc=\relax \let\PY@ff=\relax}
\def\PY@tok#1{\csname PY@tok@#1\endcsname}
\def\PY@toks#1+{\ifx\relax#1\empty\else%
    \PY@tok{#1}\expandafter\PY@toks\fi}
\def\PY@do#1{\PY@bc{\PY@tc{\PY@ul{%
    \PY@it{\PY@bf{\PY@ff{#1}}}}}}}
\def\PY#1#2{\PY@reset\PY@toks#1+\relax+\PY@do{#2}}

\@namedef{PY@tok@w}{\def\PY@tc##1{\textcolor[rgb]{0.73,0.73,0.73}{##1}}}
\@namedef{PY@tok@c}{\let\PY@it=\textit\def\PY@tc##1{\textcolor[rgb]{0.24,0.48,0.48}{##1}}}
\@namedef{PY@tok@cp}{\def\PY@tc##1{\textcolor[rgb]{0.61,0.40,0.00}{##1}}}
\@namedef{PY@tok@k}{\let\PY@bf=\textbf\def\PY@tc##1{\textcolor[rgb]{0.00,0.50,0.00}{##1}}}
\@namedef{PY@tok@kp}{\def\PY@tc##1{\textcolor[rgb]{0.00,0.50,0.00}{##1}}}
\@namedef{PY@tok@kt}{\def\PY@tc##1{\textcolor[rgb]{0.69,0.00,0.25}{##1}}}
\@namedef{PY@tok@o}{\def\PY@tc##1{\textcolor[rgb]{0.40,0.40,0.40}{##1}}}
\@namedef{PY@tok@ow}{\let\PY@bf=\textbf\def\PY@tc##1{\textcolor[rgb]{0.67,0.13,1.00}{##1}}}
\@namedef{PY@tok@nb}{\def\PY@tc##1{\textcolor[rgb]{0.00,0.50,0.00}{##1}}}
\@namedef{PY@tok@nf}{\def\PY@tc##1{\textcolor[rgb]{0.00,0.00,1.00}{##1}}}
\@namedef{PY@tok@nc}{\let\PY@bf=\textbf\def\PY@tc##1{\textcolor[rgb]{0.00,0.00,1.00}{##1}}}
\@namedef{PY@tok@nn}{\let\PY@bf=\textbf\def\PY@tc##1{\textcolor[rgb]{0.00,0.00,1.00}{##1}}}
\@namedef{PY@tok@ne}{\let\PY@bf=\textbf\def\PY@tc##1{\textcolor[rgb]{0.80,0.25,0.22}{##1}}}
\@namedef{PY@tok@nv}{\def\PY@tc##1{\textcolor[rgb]{0.10,0.09,0.49}{##1}}}
\@namedef{PY@tok@no}{\def\PY@tc##1{\textcolor[rgb]{0.53,0.00,0.00}{##1}}}
\@namedef{PY@tok@nl}{\def\PY@tc##1{\textcolor[rgb]{0.46,0.46,0.00}{##1}}}
\@namedef{PY@tok@ni}{\let\PY@bf=\textbf\def\PY@tc##1{\textcolor[rgb]{0.44,0.44,0.44}{##1}}}
\@namedef{PY@tok@na}{\def\PY@tc##1{\textcolor[rgb]{0.41,0.47,0.13}{##1}}}
\@namedef{PY@tok@nt}{\let\PY@bf=\textbf\def\PY@tc##1{\textcolor[rgb]{0.00,0.50,0.00}{##1}}}
\@namedef{PY@tok@nd}{\def\PY@tc##1{\textcolor[rgb]{0.67,0.13,1.00}{##1}}}
\@namedef{PY@tok@s}{\def\PY@tc##1{\textcolor[rgb]{0.73,0.13,0.13}{##1}}}
\@namedef{PY@tok@sd}{\let\PY@it=\textit\def\PY@tc##1{\textcolor[rgb]{0.73,0.13,0.13}{##1}}}
\@namedef{PY@tok@si}{\let\PY@bf=\textbf\def\PY@tc##1{\textcolor[rgb]{0.64,0.35,0.47}{##1}}}
\@namedef{PY@tok@se}{\let\PY@bf=\textbf\def\PY@tc##1{\textcolor[rgb]{0.67,0.36,0.12}{##1}}}
\@namedef{PY@tok@sr}{\def\PY@tc##1{\textcolor[rgb]{0.64,0.35,0.47}{##1}}}
\@namedef{PY@tok@ss}{\def\PY@tc##1{\textcolor[rgb]{0.10,0.09,0.49}{##1}}}
\@namedef{PY@tok@sx}{\def\PY@tc##1{\textcolor[rgb]{0.00,0.50,0.00}{##1}}}
\@namedef{PY@tok@m}{\def\PY@tc##1{\textcolor[rgb]{0.40,0.40,0.40}{##1}}}
\@namedef{PY@tok@gh}{\let\PY@bf=\textbf\def\PY@tc##1{\textcolor[rgb]{0.00,0.00,0.50}{##1}}}
\@namedef{PY@tok@gu}{\let\PY@bf=\textbf\def\PY@tc##1{\textcolor[rgb]{0.50,0.00,0.50}{##1}}}
\@namedef{PY@tok@gd}{\def\PY@tc##1{\textcolor[rgb]{0.63,0.00,0.00}{##1}}}
\@namedef{PY@tok@gi}{\def\PY@tc##1{\textcolor[rgb]{0.00,0.52,0.00}{##1}}}
\@namedef{PY@tok@gr}{\def\PY@tc##1{\textcolor[rgb]{0.89,0.00,0.00}{##1}}}
\@namedef{PY@tok@ge}{\let\PY@it=\textit}
\@namedef{PY@tok@gs}{\let\PY@bf=\textbf}
\@namedef{PY@tok@ges}{\let\PY@bf=\textbf\let\PY@it=\textit}
\@namedef{PY@tok@gp}{\let\PY@bf=\textbf\def\PY@tc##1{\textcolor[rgb]{0.00,0.00,0.50}{##1}}}
\@namedef{PY@tok@go}{\def\PY@tc##1{\textcolor[rgb]{0.44,0.44,0.44}{##1}}}
\@namedef{PY@tok@gt}{\def\PY@tc##1{\textcolor[rgb]{0.00,0.27,0.87}{##1}}}
\@namedef{PY@tok@err}{\def\PY@bc##1{{\setlength{\fboxsep}{\string -\fboxrule}\fcolorbox[rgb]{1.00,0.00,0.00}{1,1,1}{\strut ##1}}}}
\@namedef{PY@tok@kc}{\let\PY@bf=\textbf\def\PY@tc##1{\textcolor[rgb]{0.00,0.50,0.00}{##1}}}
\@namedef{PY@tok@kd}{\let\PY@bf=\textbf\def\PY@tc##1{\textcolor[rgb]{0.00,0.50,0.00}{##1}}}
\@namedef{PY@tok@kn}{\let\PY@bf=\textbf\def\PY@tc##1{\textcolor[rgb]{0.00,0.50,0.00}{##1}}}
\@namedef{PY@tok@kr}{\let\PY@bf=\textbf\def\PY@tc##1{\textcolor[rgb]{0.00,0.50,0.00}{##1}}}
\@namedef{PY@tok@bp}{\def\PY@tc##1{\textcolor[rgb]{0.00,0.50,0.00}{##1}}}
\@namedef{PY@tok@fm}{\def\PY@tc##1{\textcolor[rgb]{0.00,0.00,1.00}{##1}}}
\@namedef{PY@tok@vc}{\def\PY@tc##1{\textcolor[rgb]{0.10,0.09,0.49}{##1}}}
\@namedef{PY@tok@vg}{\def\PY@tc##1{\textcolor[rgb]{0.10,0.09,0.49}{##1}}}
\@namedef{PY@tok@vi}{\def\PY@tc##1{\textcolor[rgb]{0.10,0.09,0.49}{##1}}}
\@namedef{PY@tok@vm}{\def\PY@tc##1{\textcolor[rgb]{0.10,0.09,0.49}{##1}}}
\@namedef{PY@tok@sa}{\def\PY@tc##1{\textcolor[rgb]{0.73,0.13,0.13}{##1}}}
\@namedef{PY@tok@sb}{\def\PY@tc##1{\textcolor[rgb]{0.73,0.13,0.13}{##1}}}
\@namedef{PY@tok@sc}{\def\PY@tc##1{\textcolor[rgb]{0.73,0.13,0.13}{##1}}}
\@namedef{PY@tok@dl}{\def\PY@tc##1{\textcolor[rgb]{0.73,0.13,0.13}{##1}}}
\@namedef{PY@tok@s2}{\def\PY@tc##1{\textcolor[rgb]{0.73,0.13,0.13}{##1}}}
\@namedef{PY@tok@sh}{\def\PY@tc##1{\textcolor[rgb]{0.73,0.13,0.13}{##1}}}
\@namedef{PY@tok@s1}{\def\PY@tc##1{\textcolor[rgb]{0.73,0.13,0.13}{##1}}}
\@namedef{PY@tok@mb}{\def\PY@tc##1{\textcolor[rgb]{0.40,0.40,0.40}{##1}}}
\@namedef{PY@tok@mf}{\def\PY@tc##1{\textcolor[rgb]{0.40,0.40,0.40}{##1}}}
\@namedef{PY@tok@mh}{\def\PY@tc##1{\textcolor[rgb]{0.40,0.40,0.40}{##1}}}
\@namedef{PY@tok@mi}{\def\PY@tc##1{\textcolor[rgb]{0.40,0.40,0.40}{##1}}}
\@namedef{PY@tok@il}{\def\PY@tc##1{\textcolor[rgb]{0.40,0.40,0.40}{##1}}}
\@namedef{PY@tok@mo}{\def\PY@tc##1{\textcolor[rgb]{0.40,0.40,0.40}{##1}}}
\@namedef{PY@tok@ch}{\let\PY@it=\textit\def\PY@tc##1{\textcolor[rgb]{0.24,0.48,0.48}{##1}}}
\@namedef{PY@tok@cm}{\let\PY@it=\textit\def\PY@tc##1{\textcolor[rgb]{0.24,0.48,0.48}{##1}}}
\@namedef{PY@tok@cpf}{\let\PY@it=\textit\def\PY@tc##1{\textcolor[rgb]{0.24,0.48,0.48}{##1}}}
\@namedef{PY@tok@c1}{\let\PY@it=\textit\def\PY@tc##1{\textcolor[rgb]{0.24,0.48,0.48}{##1}}}
\@namedef{PY@tok@cs}{\let\PY@it=\textit\def\PY@tc##1{\textcolor[rgb]{0.24,0.48,0.48}{##1}}}

\def\PYZbs{\char`\\}
\def\PYZus{\char`\_}
\def\PYZob{\char`\{}
\def\PYZcb{\char`\}}
\def\PYZca{\char`\^}
\def\PYZam{\char`\&}
\def\PYZlt{\char`\<}
\def\PYZgt{\char`\>}
\def\PYZsh{\char`\#}
\def\PYZpc{\char`\%}
\def\PYZdl{\char`\$}
\def\PYZhy{\char`\-}
\def\PYZsq{\char`\'}
\def\PYZdq{\char`\"}
\def\PYZti{\char`\~}
% for compatibility with earlier versions
\def\PYZat{@}
\def\PYZlb{[}
\def\PYZrb{]}
\makeatother


    % For linebreaks inside Verbatim environment from package fancyvrb.
    \makeatletter
        \newbox\Wrappedcontinuationbox
        \newbox\Wrappedvisiblespacebox
        \newcommand*\Wrappedvisiblespace {\textcolor{red}{\textvisiblespace}}
        \newcommand*\Wrappedcontinuationsymbol {\textcolor{red}{\llap{\tiny$\m@th\hookrightarrow$}}}
        \newcommand*\Wrappedcontinuationindent {3ex }
        \newcommand*\Wrappedafterbreak {\kern\Wrappedcontinuationindent\copy\Wrappedcontinuationbox}
        % Take advantage of the already applied Pygments mark-up to insert
        % potential linebreaks for TeX processing.
        %        {, <, #, %, $, ' and ": go to next line.
        %        _, }, ^, &, >, - and ~: stay at end of broken line.
        % Use of \textquotesingle for straight quote.
        \newcommand*\Wrappedbreaksatspecials {%
            \def\PYGZus{\discretionary{\char`\_}{\Wrappedafterbreak}{\char`\_}}%
            \def\PYGZob{\discretionary{}{\Wrappedafterbreak\char`\{}{\char`\{}}%
            \def\PYGZcb{\discretionary{\char`\}}{\Wrappedafterbreak}{\char`\}}}%
            \def\PYGZca{\discretionary{\char`\^}{\Wrappedafterbreak}{\char`\^}}%
            \def\PYGZam{\discretionary{\char`\&}{\Wrappedafterbreak}{\char`\&}}%
            \def\PYGZlt{\discretionary{}{\Wrappedafterbreak\char`\<}{\char`\<}}%
            \def\PYGZgt{\discretionary{\char`\>}{\Wrappedafterbreak}{\char`\>}}%
            \def\PYGZsh{\discretionary{}{\Wrappedafterbreak\char`\#}{\char`\#}}%
            \def\PYGZpc{\discretionary{}{\Wrappedafterbreak\char`\%}{\char`\%}}%
            \def\PYGZdl{\discretionary{}{\Wrappedafterbreak\char`\$}{\char`\$}}%
            \def\PYGZhy{\discretionary{\char`\-}{\Wrappedafterbreak}{\char`\-}}%
            \def\PYGZsq{\discretionary{}{\Wrappedafterbreak\textquotesingle}{\textquotesingle}}%
            \def\PYGZdq{\discretionary{}{\Wrappedafterbreak\char`\"}{\char`\"}}%
            \def\PYGZti{\discretionary{\char`\~}{\Wrappedafterbreak}{\char`\~}}%
        }
        % Some characters . , ; ? ! / are not pygmentized.
        % This macro makes them "active" and they will insert potential linebreaks
        \newcommand*\Wrappedbreaksatpunct {%
            \lccode`\~`\.\lowercase{\def~}{\discretionary{\hbox{\char`\.}}{\Wrappedafterbreak}{\hbox{\char`\.}}}%
            \lccode`\~`\,\lowercase{\def~}{\discretionary{\hbox{\char`\,}}{\Wrappedafterbreak}{\hbox{\char`\,}}}%
            \lccode`\~`\;\lowercase{\def~}{\discretionary{\hbox{\char`\;}}{\Wrappedafterbreak}{\hbox{\char`\;}}}%
            \lccode`\~`\:\lowercase{\def~}{\discretionary{\hbox{\char`\:}}{\Wrappedafterbreak}{\hbox{\char`\:}}}%
            \lccode`\~`\?\lowercase{\def~}{\discretionary{\hbox{\char`\?}}{\Wrappedafterbreak}{\hbox{\char`\?}}}%
            \lccode`\~`\!\lowercase{\def~}{\discretionary{\hbox{\char`\!}}{\Wrappedafterbreak}{\hbox{\char`\!}}}%
            \lccode`\~`\/\lowercase{\def~}{\discretionary{\hbox{\char`\/}}{\Wrappedafterbreak}{\hbox{\char`\/}}}%
            \catcode`\.\active
            \catcode`\,\active
            \catcode`\;\active
            \catcode`\:\active
            \catcode`\?\active
            \catcode`\!\active
            \catcode`\/\active
            \lccode`\~`\~
        }
    \makeatother

    \let\OriginalVerbatim=\Verbatim
    \makeatletter
    \renewcommand{\Verbatim}[1][1]{%
        %\parskip\z@skip
        \sbox\Wrappedcontinuationbox {\Wrappedcontinuationsymbol}%
        \sbox\Wrappedvisiblespacebox {\FV@SetupFont\Wrappedvisiblespace}%
        \def\FancyVerbFormatLine ##1{\hsize\linewidth
            \vtop{\raggedright\hyphenpenalty\z@\exhyphenpenalty\z@
                \doublehyphendemerits\z@\finalhyphendemerits\z@
                \strut ##1\strut}%
        }%
        % If the linebreak is at a space, the latter will be displayed as visible
        % space at end of first line, and a continuation symbol starts next line.
        % Stretch/shrink are however usually zero for typewriter font.
        \def\FV@Space {%
            \nobreak\hskip\z@ plus\fontdimen3\font minus\fontdimen4\font
            \discretionary{\copy\Wrappedvisiblespacebox}{\Wrappedafterbreak}
            {\kern\fontdimen2\font}%
        }%

        % Allow breaks at special characters using \PYG... macros.
        \Wrappedbreaksatspecials
        % Breaks at punctuation characters . , ; ? ! and / need catcode=\active
        \OriginalVerbatim[#1,codes*=\Wrappedbreaksatpunct]%
    }
    \makeatother

    % Exact colors from NB
    \definecolor{incolor}{HTML}{303F9F}
    \definecolor{outcolor}{HTML}{D84315}
    \definecolor{cellborder}{HTML}{CFCFCF}
    \definecolor{cellbackground}{HTML}{F7F7F7}

    % prompt
    \makeatletter
    \newcommand{\boxspacing}{\kern\kvtcb@left@rule\kern\kvtcb@boxsep}
    \makeatother
    \newcommand{\prompt}[4]{
        {\ttfamily\llap{{\color{#2}[#3]:\hspace{3pt}#4}}\vspace{-\baselineskip}}
    }
    

    
    % Prevent overflowing lines due to hard-to-break entities
    \sloppy
    % Setup hyperref package
    \hypersetup{
      breaklinks=true,  % so long urls are correctly broken across lines
      colorlinks=true,
      urlcolor=urlcolor,
      linkcolor=linkcolor,
      citecolor=citecolor,
      }
    % Slightly bigger margins than the latex defaults
    
    \geometry{verbose,tmargin=1in,bmargin=1in,lmargin=1in,rmargin=1in}
    
    

\begin{document}
    
    \maketitle
    
    

    
    \section{Social Finance (online appendix and
notes)}\label{social-finance-online-appendix-and-notes}

\textbf{Jonathan Conning}, Department of Economics, Hunter College and
The Graduate Center, City University of New York

\textbf{Jonathan Morduch}, NYU Wagner Graduate School of Public Service

    \emph{This is a Jupyter Notebook to accompany the paper that offers a
summary of main arguments and python code for simulations/visualizations
in the paper.}

    \textbf{Abstract:} We propose a framework for understanding how social
investors' who seek to maximize a combination of private and social
returns from investments in a portfolio of new and established
microfinance institutions. The model takes into account the endogeneity
of loan contract terms as well as the capital structure of the financial
institutions that may emerge to serve target groups of borrowers
differentiated primarily by their levels of initial average net worth,
and how social investments might transform those patterns. We build upon
one of the workhorse models of modern corporate finance (Tirole, 2007)
which features limited liability, multiple layers of moral hazard and
costly monitoring to explain patterns of financial intermediation, and
in our framework, the role and modes of social investment. We pinpoint
the role of the subsidies and guarantees implicit in social investors'
equity and quasi-equity investments and the role they play in attracting
private capital investors and sustaining productivity-enhancing
financial intermediation that might otherwise not have taken place.

    \section{Introduction}\label{introduction}

\ldots We tackle the logic and tensions inherent in social finance---the
support, with philanthropic objectives, of nonprofits, social businesses
like the Grameen Bank, and profit-maximizing businesses serving the
poor. The principles behind the new world of philanthropy and social
action have not been well-explored by economists. We develop a theory of
``social finance'' to parallel the modern theory of corporate
finance\ldots{}

    We extend a model of capital constraints and financial intermediation
with active monitors similar to Holmstrom and Tirole (1997), Conning
(1999), but extends the model to focus on how bank capital structure
varies across banks depending on the monitoring-intensity of their loan
portfolio (determined in turn by the average net worth of its borrowers)
and the role that social investors may play in creating and expanding
loan access via structured finance.

This model itself is built upon a simple model of credit rationing due
to borrower moral hazard and limited liability, the `workhorse' model of
Tirole's (2006) \emph{The Theory of Corporate Finance}. Risk-neutral
entrepreneurs have access to an investment project which requires a
lump-sum investment \(I\) to get started, but they do not have liquid
funds so they seek to borrow the entire amount from financial
intermediaries. The problem of moral hazard will dictate that optimal
contracts must reward project success more highly than project failure
in order to give entrepreneurs an incentive to want to increase the
probability of success. Under many plausible parameter scenaries te
optimal contract will require that loan repayments in the failure
state(s) be met out of assets that are additional or `collateral' to the
generated project returns. Lenders will find it unprofitable to lend to
any borrower who cannot credibly pledge assets below a minimum
collateral requirement \(\underline A\). A simple graphical analysis of
this collateral based lending model is laid out in a notebook
\href{basicmodel.ipynb}{here}

Local intermediaries may be able in `active monitoring' that directly
lowers borrowers' scope for moral hazard, lowering the minimum
collateral requirements necessary to attract outside investors, thereby
expanding capital access. But monitoring is a costly activity that is
itself subject to moral hazard. For this reason an optimal contract will
require monitoring intermediaries to have enough of their own capital at
risk in a loan so as to provide incentives to appropriately monitor to
protect any outside investor's interests.

In contrast to the earlier mentioned papers, in this paper we posit the
idea that local intermediary monitoring capacity is
neighborhood-specific, and neighborhoods are largely segregated by the
average level of pledgeable assets of its residents. This leads us to a
focus on the optimal capital structure of neighborhood-specific banks
(or more broadly to the optimal capital structure of different types of
banks, depending on the monitoring intensity of their loan portfolio).

    There are up to four types of agents in the model:

\begin{enumerate}
\def\labelenumi{\arabic{enumi}.}
\tightlist
\item
  risk-neutral entrepreneur households that can run small businesses if
  they are able to cover lump sum \(I\). Household differ in terms of
  their initial pledgeable assets \(A\) (tied up in other projects but
  can be liquidated at a cost to cover obligations.
\item
  one or more local financial intermediaries in each neighborhood,
  wholly or partly owned and managed by locally informed equity
  investors. These financial institutions may lend out of their own
  equity capital. Their own at risk investment in a particular
  entreprenneur's project is labeled \(I^m\) to indicate that they
  monitor to try to limit scope for moral hazard.
\item
  private uninformed investors (possibly including savings depositors).
\item
  Finally, social investors may be able to affect the nature and depth
  of the above relationsips through additional investments, subsidies
  and guarantees of their own.
\end{enumerate}

Entrepreneurs in neighborhood \(j\) have pledgeable assets \(A_j\)
(assets that are tied up in other productive uses but could be
liquidated to pay off a loan). Depending on the characteristics of the
loan projects and the level of \(A_j\) the model generates one of four
types of lending structures:

Depending on their initial holding of \(A\) and parameters of the
problem, entrepreneurs will in the end be either: 1. not funded 2.
funded only by a non-leveraged local intermediary, so \(I = I^m\) 3.
funded by a leveraged intermediary: so \(I = I^m + I^u\)

The model can be closed so that, depending on the characteristics of
loans, the initial distribution of pledgeable assets across
neighborhoods and entrepreneurs, and the economy-wide levels of
intermediary and uninformed capital we can predict the rate of return on
uninformed and intermediary capital as well as bank capital structure
and loan terms across the population.

    \subsubsection{Model code}\label{model-code}

Many of the functions used to simulate and visualize the model below are
written up in python in a \href{socialfinance.py}{socialfinance} module.

    \begin{tcolorbox}[breakable, size=fbox, boxrule=1pt, pad at break*=1mm,colback=cellbackground, colframe=cellborder]
\prompt{In}{incolor}{1}{\boxspacing}
\begin{Verbatim}[commandchars=\\\{\}]
\PY{o}{\PYZpc{}}\PY{k}{load\PYZus{}ext} autoreload
\PY{o}{\PYZpc{}}\PY{k}{autoreload} 2
\PY{o}{\PYZpc{}}\PY{k}{reload\PYZus{}ext} autoreload
\PY{k+kn}{from} \PY{n+nn}{socialfinance} \PY{k+kn}{import} \PY{n}{Bank}
\end{Verbatim}
\end{tcolorbox}

    \begin{tcolorbox}[breakable, size=fbox, boxrule=1pt, pad at break*=1mm,colback=cellbackground, colframe=cellborder]
\prompt{In}{incolor}{2}{\boxspacing}
\begin{Verbatim}[commandchars=\\\{\}]
\PY{k+kn}{import} \PY{n+nn}{numpy} \PY{k}{as} \PY{n+nn}{np}
\PY{k+kn}{import} \PY{n+nn}{matplotlib}\PY{n+nn}{.}\PY{n+nn}{pyplot} \PY{k}{as} \PY{n+nn}{plt}
\PY{k+kn}{from} \PY{n+nn}{ipywidgets} \PY{k+kn}{import} \PY{n}{interact}
\PY{n}{plt}\PY{o}{.}\PY{n}{rcParams}\PY{p}{[}\PY{l+s+s1}{\PYZsq{}}\PY{l+s+s1}{figure.figsize}\PY{l+s+s1}{\PYZsq{}}\PY{p}{]} \PY{o}{=} \PY{p}{(}\PY{l+m+mf}{5.0}\PY{p}{,} \PY{l+m+mf}{5.0}\PY{p}{)}
\PY{n}{plt}\PY{o}{.}\PY{n}{rcParams}\PY{p}{[}\PY{l+s+s1}{\PYZsq{}}\PY{l+s+s1}{text.usetex}\PY{l+s+s1}{\PYZsq{}}\PY{p}{]} \PY{o}{=} \PY{k+kc}{True}
\end{Verbatim}
\end{tcolorbox}

    \subsection{Model parameters}\label{model-parameters}

We imagine a continuum of neighborhoods indexed by the level of
pledgeable assets \(A\) of its residents. We order these from low to
high values.

An \texttt{mfi} or microfinance institution serves the neighborhood.
Represented as a \texttt{Bank} object in the code.

    \begin{tcolorbox}[breakable, size=fbox, boxrule=1pt, pad at break*=1mm,colback=cellbackground, colframe=cellborder]
\prompt{In}{incolor}{3}{\boxspacing}
\begin{Verbatim}[commandchars=\\\{\}]
\PY{n}{A} \PY{o}{=} \PY{n}{np}\PY{o}{.}\PY{n}{linspace}\PY{p}{(}\PY{l+m+mi}{0}\PY{p}{,}\PY{l+m+mi}{120}\PY{p}{,}\PY{l+m+mi}{500}\PY{p}{)}   
\PY{n}{mfi} \PY{o}{=} \PY{n}{Bank}\PY{p}{(}\PY{n}{A}\PY{p}{,} \PY{n}{beta} \PY{o}{=} \PY{l+m+mf}{1.2}\PY{p}{)}     
\PY{n}{mfi}\PY{o}{.}\PY{n}{print\PYZus{}params}\PY{p}{(}\PY{p}{)}    
\end{Verbatim}
\end{tcolorbox}

    \begin{Verbatim}[commandchars=\\\{\}]
Amax = 140, B0 = 30, F = 0, I = 100, K = 12000, X = 200, alpha = 0.5, beta =
1.2, f = 30, gamma = 1.0, p = 0.97, q = 0.82
    \end{Verbatim}

    \textbf{Intermediary fixed cost per borrower}

The model allows for the possibility of fixed costs per loan \(f\) and
fixed costs per bank \(F\), but we'll start by setting these fixed costs
to zero.

    \textbf{Monitoring technology:} We assume a simple linear relationship
betweeen monitoring intensity \(m\) (=monitoring expense) and the extent
of moral hazard as captured by the private benefits \(B(m)\) the client
stands to capture from non-diligence.

    \[B(m) = B_0 - \alpha \cdot m\]

    We don't need to assume a linear relationship -- which implies a
constant marginal cost to monitoring -- we could imagine that monitoring
is at first falling and then rising marginal cost. The linear assumption
just helps to make the results slightly more stark and easier to derive.

    \subsubsection{Neighborhoods, Entrepreneurs and
assets}\label{neighborhoods-entrepreneurs-and-assets}

There are \(J\) neighborhoods with \(N\) enterepreneurs per
neighborhood. The neighborhoods are segregated by pledgeable assets or
wealth (and to simplify, everyone in a particular neighborhood has the
same level of assets as all their neighbors).

We assume a uniform distribution across neighborhoods. A household: - in
the poorest (\(j=0\)) neighborhood has pledgeable assets \(A=0\) - in
the richest neighborhood has \(A=A^{max}\) - in neighborhood \(j\) has
pledgeable assets \(A_j = j \cdot \frac{A^{max}}{J}\)

    Each of the \(N\) would-be entrepreneurs in each of the \(J\)
neighborhoods has access to a project that, for lump sum investment
\(I\), produces expected return \(p\cdot X_s\). Total ``potential
demand,'' if all projects were funded, is therefore:
\(\bar{K} = N \cdot J \cdot I\)

We assume however that there is not enough local intermediary monitoring
capital to satisfy total demand in the neighborhood. This means
financial intermediaries must leverate the remaining funds from a larger
market for uninformed capital. Uninformed capital investors will
participate as long as they can expect to earn the opportunity cost of
funds \(r\).

    \subsubsection{Minimum collateral
requirements}\label{minimum-collateral-requirements}

An entrepreneur's project generates a return \(X_s\) with probability
\(p\) and a return \(X_f\) with probability \(1-p\). A financial
contract divides the project returns \(X_i\) between returns to the
monitoring intermediary \(R_i\), returns to the entrepreneur \(s_i\),
and the uninformed lender gets what's left or \(X_i-s_i-R_i\).

    \paragraph{\texorpdfstring{Minimum collateral for a no-monitoring lender
(\(m=0\))}{Minimum collateral for a no-monitoring lender (m=0)}}\label{minimum-collateral-for-a-no-monitoring-lender-m0}

An uninformed lender cannot observe whether the entrepreneur has been
diligent (chose project \(p\)) or not (chose project \(q\) and got
private benefit \(B(0)\)). They will insist the contract provide
incentives to diligence. That is the contract must satisfy the borrowers
incentive compatibility constraint:

\[E(s|p) \ge E(s|q) + B(0)\]

Expanding and regrouping terms we can express this as:

\[s_s  \ge s_f + \frac{B(0)}{p-q}\]

Which is to say that the reward for success must be sufficiently greater
than the reward to failure, to provide the entrepreneur with an
incentive to choose the high return project.

If limited liability puts a lower bound on how small of \(s_f=-A\)
(where \(A\) are the entrepreneur's pledgeable assets) then this implies
a borrower with pledgeable assets \(A\) earns a `limited liability' rent
of:

    \[E(s|p)= -A + p\frac{B(0)}{p-q}\]

    Less well-off entrepreneurs with smaller \(A\) must earn higher rents to
maintain incentives. Intuitively, since limited liability limits how
much they can be punished for failure outcomes, incentives can only be
maintained by keeping repayments low in sucess states, but this is
costly to the lender, since it means they cannot collect too much in the
success state (demand too high of a repayment) without destroying the
incentive to choose the high return project.

The lender must be able to cover this rent and her opportunity cost of
funds \(\gamma I\) and fixed costs per loan \(F\) (we assume the latter
is paid in the second period, hence not discounted). For now we indicate
fixed costs simply as \(F\) but note that we'll later break this into
fixed costs at the loan level (e.g.~minimum cost of processing a loan,
regardless of organization size) and organization-level fixed costs
(average fixed costs will decline with organization size).

So a lender will just be able to break even and be willing to
participate when the poorest borrower has assets :

\[
E(x|p) - E(s|p) = \gamma \cdot I + F
\]

\[
E(x|p)  + A  - p\frac{B(0)}{p-q} = \gamma \cdot I + F
\]

Solving for the \(A\) as a function of \(m\) where this exactly holds
gives us the \textbf{minimum collateral requirement for a no-monitoring
or uninformed lender} who has opportunity cost of funds \(\gamma\) but
cannot monitor.

\[
\underline A^u(0) = \frac{p \cdot B(0)}{p-q} 
 - \left[ {pX - \gamma I -  F} \right]
 \]

If a borrower has pledgeable assets \(A\) in excess of the minimum
collateral required by an uninformed lender \(\underline A(0,N)\) then
they'll pledge \(\underline A(0,N)\) and borrow entirely from the
uninformed lender, so \(I = I^u\) and the cost of funds to the borrower
in this competitive environment will be \(\gamma\)

Entrepreneurs with \(A \lt \underline A(0,N)\) have no choice but to try
to borrow via a more expensive monitoring local intermediary.

    For our particular example:

    \begin{tcolorbox}[breakable, size=fbox, boxrule=1pt, pad at break*=1mm,colback=cellbackground, colframe=cellborder]
\prompt{In}{incolor}{4}{\boxspacing}
\begin{Verbatim}[commandchars=\\\{\}]
\PY{l+s+sa}{f}\PY{l+s+s1}{\PYZsq{}}\PY{l+s+s1}{Minimum collateral requirement for a non\PYZhy{}monitored loan AM(0, 1) = }\PY{l+s+si}{\PYZob{}}\PY{n}{mfi}\PY{o}{.}\PY{n}{AM}\PY{p}{(}\PY{l+m+mi}{0}\PY{p}{)}\PY{l+s+si}{:}\PY{l+s+s1}{.2f}\PY{l+s+si}{\PYZcb{}}\PY{l+s+s1}{\PYZsq{}}
\end{Verbatim}
\end{tcolorbox}

            \begin{tcolorbox}[breakable, size=fbox, boxrule=.5pt, pad at break*=1mm, opacityfill=0]
\prompt{Out}{outcolor}{4}{\boxspacing}
\begin{Verbatim}[commandchars=\\\{\}]
'Minimum collateral requirement for a non-monitored loan AM(0, 1) = 130.00'
\end{Verbatim}
\end{tcolorbox}
        
    This may seem high -- it would seem the bank is asking for \$97
collateral for a \$100 loan, but `pledgeable assets' might involve
partially illiquid assets (e.g.~a sofa, a vehicle, or wages that can be
garnished) that borrowers in more affluent neighborhoods are likely to
have. As we shall see shortly a borrower who can pledge this much can
expect to make a very good return from this relatively low cost loan.

It's the people that cannot pledge this amount who suffer by being
denied these relatively low-cost (zero-monitoring loans) who must
instead accept higher cost monitoring-intensive loans, or perhaps be
denied any type of loan.

    \paragraph{Minimum collateral for a monitoring
lender}\label{minimum-collateral-for-a-monitoring-lender}

A monitoring intermediary has the advantage of being able to monitor to
directly lower the scope for moral hazard via \(B(m)\) but they charge a
higher cost of funds because (a) they face a higher opportunity cost of
funds \(\beta > \gamma\) and because (b) they must also be compensated
for the cost of monitoring \(m\).

Monitoring lowers the private benefit from non-diligence:

\[E(s|p)= -A + p\frac{B(m)}{p-q}\]

but monitoring adds a cost that must now be paid for:

\[E(x|p) - E(s|p) = \beta \cdot I + m + F\]

The expression below shows how monitoring intensity \(m\) can lower the
collateral requirement for a monitoring intermediary when they are the
only lender:

    \textbf{Non-leveraged or Equity-only MFI}

\[
\underline A^e(m) = \frac{p \cdot B(m)}{p-q} - \left[ {pX - \beta I - F} \right]  + m 
\]

Note that as monitoring \(m\) must be paid for out of available project
surplus, which reduces what is left to the entrepreneur after making all
necessary repayments. Competition in a competitive market will lead to
contracts with the minimum required monitoring, to keep costs to the
borrower down. That is they'll choose a monitoring intensity \(m=m(A)\)
that brings the monitoring minimum collateral requirement down to match
the borrower's available pledgeable assets \(A\) and make the loan
feasible:

\[\underline A^e(m(A)) = A\]

The derivation of the closed form solutions for \(m(A)\) is found below.

    \textbf{Leveraged MFI}

If the local intermediary MFI can leverage outside capital it can
potentially substitute cheaper outside financing for more expensive
local intermediary (equity) financing.

An outside lender will however only participate in a financing structure
if it can be sure the monitoring intermediary has enough `skin in the
game' to have incentives to carry out the unobservable monitoring on the
loan at this minimum intensity required for expected repayments to cover
the uninformed lender's costs.

A contract allocates claims as \(s_i\) to the entrepreneur, \(R_i\) to
the monitoring lender and \(X_i -R_i - s_i\) to the uninformed lender
where \(i = S, F\). The monitor's incentive compatibility constraint
requires that they earn more from being diligent in monitoring at
expense \(m\) than from not monitoring:

\[R_s \ge R_f + \frac{m}{p-q} \]

    The lowest cost way to satisfy the monitor's incentive constraint (to
have it bind) implies leaving a monitoring rent of:

\[E(R|p) = R_f + p \cdot \frac{m}{p-q}\]

to the intermediary. This is analogous to the limited liability rent to
the borrower.

\textbf{Scarce Intermediary Capital}

Suppose there is only on intermediary in the neighborhood. Then they
will put nothing at risk (\(R_f=0\)) and earn an economic rent of

\[
p \frac{m}{\Delta p}
\]

Now the

\[
\underline A^M(m) = \frac{p \cdot B(m)}{p-q} - \left[ {pX - \beta I - F} \right]  +  p \frac{m}{p-q}
\]

    \textbf{Competition for intermediary services}

Assume instead now there is free entry into intermediation services. in
the neighborhood. Intermediaries will now compete to lend to borrowers
and leverage outside funds from uninformed lenders by putting up capital
of their own. This is ``skin in the game'' but it also reduces the rent.

Competition will insure that the untermediary earns zero profits
(alternatively, we could assume intermediaries have a social mission to
reach as many entrepreneurs as possible, and earn zero profits). If the
intermediary puts up \(I^m\) of each \(I\) loan and takes first losses
then they stand to capture \(R_f = -\beta I^m\) if the project fails and
\(R_s = -\beta I^m + \frac{m}{p-q}\) if the project succeeds.
Competition drives intermediary profits down to zero. The expected
returns from the contract (left hand side below) must cover the cost of
monitoring \(m\) plus the fixed cost per borrower:

\$\$p \cdot \frac{m}{p-q} \ge \beta I\^{}m + m

    We can use this to find the size of the stake (or skin in the game)
\(I^m\) the monitoring intermediary must have in the project in order to
have incentive to monitor:

\[
I^m = \frac{1}{\beta}\frac{q \cdot m}{p-q} 
\]

    which is rising with the required amount of monitoring \(m\) (to be
determined).

The uninformed lender puts up \(I^u = I - I^m\) and the monitor puts up
\(I^m\). The `poorest' entrepreneur (i.e.~the one with the lowest level
of pledgeable assets A that can be reached) will be determined by the
contract where what is left of expected project returns after paying the
borrower the limited liability rent (required to make sure they choose
the high probability of success project) is just enough to pay off both
the uniformed and the informed lenders:

\[
E[X|p] - E[s|p] \ge \gamma (I - I_m) + p \frac{m}{p-q} + F
\]

rearranging and using our earlier finding that
\(\beta I^m + m = p \frac{m}{p-q}\) we can solve for the minimum
collateral requirement for the leveraged MFI:

\[
pX + A - p\frac{B(m)}{p-q} = \gamma (I - I^m) + \beta I^m +  f + m
\]

solving for \(A\) as a function of \(m\):

    \[
\underline A (m) = p \cdot \frac{B(m)}{p-q} 
- \left[ {pX - \gamma I} \right] 
+ \frac{\beta - \gamma}{\beta} \left( \frac{q \cdot m}{p-q} \right ) + m + \gamma F
\]

If \(\beta=\gamma\) this collapses to the no-leverage minimum collateral
requirement above.

If \(\beta \gt \gamma\) the premium that must be paid for intermediary
capital (even though intermediaries are earning zero rents in this
activity) adds to the cost.

    \subsection{Optimal (minimum required)
monitoring}\label{optimal-minimum-required-monitoring}

Monitoring is costly so only so much will be used as needed to lower the
minimum collateral requirement to the entrepreneur's pledgeable asset
level \(A\)\textgreater{}

    The minimum collateral requirement above can be rewritten as a function
where higher monitoring \(m\) reduces the collateral requirement from
the no-monitoring collateral level. Using the fact that
\(B(m) = B(0) - \alpha m\) we can write the minimum collateral
requirement as a function of monitoring intensity \(m\) as:

\[\underline A (m) = \underline A (0) 
  - m \left [ p \cdot \frac{\alpha}{p-q} - \frac{(\beta - \gamma)}{\beta} \left( \frac{q}{p-q} \right ) -1 \right ]\]

    Which means we can solve \(\underline A (m) = A\) for \emph{the minimum
monitoring intensity} for a borrower with pledgeable assets \(A\):

\[m(A) = \left[ \underline A(0)  - A \right] \cdot
    \frac{\beta(p-q)}{\beta p(\alpha - 1)+\gamma q}\]

    If the entrepreneus can only access an equity-only lender
(i.e.~\(I^m=I\)): {[}CHECK THIS{]}

\[\underline A^e(m) = A\]

where
\(\underline A^e(m) =  p \cdot \frac{B(m)}{p-q} - \left[ {pX - \beta I} \right] + m + \beta F\)

Solve for \(m\) to get:

\[m^e(A) = \left[ \underline A^e(0)  - A \right] \cdot
    \frac{(p-q)}{ p(\alpha-1)+ q}\]

And when \(\beta=\gamma\) these two expressions become identical.

    In this competitive setting the lender(s) earns zero profits. But the
borrower will only participate if they can expect a positive return
(borrower participation constraint). This determines the maximum
feasible level of monitoring.

\[
p \cdot X - \beta \cdot I -  F - m \ge 0
\]

\[m^{max} = p  \cdot X -\gamma I - F\]

    The two lines cross at \$\underbar A(m, \beta) = AME(m, β) or at:

\[\bar m =  \frac{\beta(I+F)(p-q)}{q}\]

So any entrepreneur with \(A<AM(\bar m)\) would be in an equity-only
loan.

    At zero monitoring it's obviously cheaper to use uninformed capital
which has lower cost \(\gamma\) rather than borrow from an local
intermediary capital which has opportunity cost \(\beta\). Since the
latter type of loans are more expensive, they'll also be associated with
higher minimum collateral requirements. Suppose
\(\beta = 1.2 \cdot \gamma\), then:

    \begin{tcolorbox}[breakable, size=fbox, boxrule=1pt, pad at break*=1mm,colback=cellbackground, colframe=cellborder]
\prompt{In}{incolor}{5}{\boxspacing}
\begin{Verbatim}[commandchars=\\\{\}]
\PY{n}{mfi}\PY{o}{.}\PY{n}{print\PYZus{}params}\PY{p}{(}\PY{p}{)}
\end{Verbatim}
\end{tcolorbox}

    \begin{Verbatim}[commandchars=\\\{\}]
Amax = 140, B0 = 30, F = 0, I = 100, K = 12000, X = 200, alpha = 0.5, beta =
1.2, f = 30, gamma = 1.0, p = 0.97, q = 0.82
    \end{Verbatim}

    \begin{tcolorbox}[breakable, size=fbox, boxrule=1pt, pad at break*=1mm,colback=cellbackground, colframe=cellborder]
\prompt{In}{incolor}{6}{\boxspacing}
\begin{Verbatim}[commandchars=\\\{\}]
\PY{n}{mfi}\PY{o}{.}\PY{n}{beta} \PY{o}{=} \PY{l+m+mf}{1.2}
\PY{n+nb}{print}\PY{p}{(}\PY{l+s+s1}{\PYZsq{}}\PY{l+s+s1}{Ame(0) = }\PY{l+s+si}{\PYZob{}:5.1f\PYZcb{}}\PY{l+s+s1}{  Am(0) = }\PY{l+s+si}{\PYZob{}:5.1f\PYZcb{}}\PY{l+s+s1}{ }\PY{l+s+s1}{\PYZsq{}}\PY{o}{.}\PY{n}{format}\PY{p}{(}\PY{n}{mfi}\PY{o}{.}\PY{n}{AMe}\PY{p}{(}\PY{l+m+mi}{0}\PY{p}{)}\PY{p}{,} \PY{n}{mfi}\PY{o}{.}\PY{n}{AM}\PY{p}{(}\PY{l+m+mi}{0}\PY{p}{)}\PY{p}{)}\PY{p}{)}
\PY{n+nb}{print}\PY{p}{(}\PY{l+s+s1}{\PYZsq{}}\PY{l+s+s1}{mcross = }\PY{l+s+si}{\PYZob{}:5.1f\PYZcb{}}\PY{l+s+s1}{   mmax = }\PY{l+s+si}{\PYZob{}:5.1f\PYZcb{}}\PY{l+s+s1}{   Amin = }\PY{l+s+si}{\PYZob{}:5.1f\PYZcb{}}\PY{l+s+s1}{\PYZsq{}}\PY{o}{.}\PY{n}{format}\PY{p}{(}\PY{n}{mfi}\PY{o}{.}\PY{n}{mcross}\PY{p}{(}\PY{p}{)}\PY{p}{,}\PY{n}{mfi}\PY{o}{.}\PY{n}{mmax}\PY{p}{(}\PY{p}{)}\PY{p}{,} \PY{n}{mfi}\PY{o}{.}\PY{n}{Amin}\PY{p}{(}\PY{p}{)}\PY{p}{)}\PY{p}{)}
\end{Verbatim}
\end{tcolorbox}

    \begin{Verbatim}[commandchars=\\\{\}]
Ame(0) = 150.0  Am(0) = 130.0
mcross =  22.0   mmax =  44.0   Amin =  51.7
    \end{Verbatim}

    \begin{tcolorbox}[breakable, size=fbox, boxrule=1pt, pad at break*=1mm,colback=cellbackground, colframe=cellborder]
\prompt{In}{incolor}{7}{\boxspacing}
\begin{Verbatim}[commandchars=\\\{\}]
\PY{n}{mfi}\PY{o}{.}\PY{n}{mmax}\PY{p}{(}\PY{p}{)}
\end{Verbatim}
\end{tcolorbox}

            \begin{tcolorbox}[breakable, size=fbox, boxrule=.5pt, pad at break*=1mm, opacityfill=0]
\prompt{Out}{outcolor}{7}{\boxspacing}
\begin{Verbatim}[commandchars=\\\{\}]
44.0
\end{Verbatim}
\end{tcolorbox}
        
    \begin{tcolorbox}[breakable, size=fbox, boxrule=1pt, pad at break*=1mm,colback=cellbackground, colframe=cellborder]
\prompt{In}{incolor}{8}{\boxspacing}
\begin{Verbatim}[commandchars=\\\{\}]
\PY{n}{mfi}\PY{o}{.}\PY{n}{plotA}\PY{p}{(}\PY{p}{)}
\end{Verbatim}
\end{tcolorbox}

    \begin{center}
    \adjustimage{max size={0.9\linewidth}{0.9\paperheight}}{socfin_m_files/socfin_m_42_0.png}
    \end{center}
    { \hspace*{\fill} \\}
    
    (NOTE the following expression assume F= 0) For any entrepreneur with
pledgeable assets \(A\) we can find optimal (minimum) amount of
monitoring):

\[\underline A^e(m) = A\] solve for \(m\) to get:

\[m(A) = \left[ \underline A^e(0)  - A \right] \cdot
    \frac{(p-q)}{ p(\alpha-1)+ q}\]

    \begin{tcolorbox}[breakable, size=fbox, boxrule=1pt, pad at break*=1mm,colback=cellbackground, colframe=cellborder]
\prompt{In}{incolor}{9}{\boxspacing}
\begin{Verbatim}[commandchars=\\\{\}]
\PY{n}{mfi}\PY{o}{.}\PY{n}{mcross}\PY{p}{(}\PY{p}{)}
\end{Verbatim}
\end{tcolorbox}

            \begin{tcolorbox}[breakable, size=fbox, boxrule=.5pt, pad at break*=1mm, opacityfill=0]
\prompt{Out}{outcolor}{9}{\boxspacing}
\begin{Verbatim}[commandchars=\\\{\}]
21.951219512195127
\end{Verbatim}
\end{tcolorbox}
        
    \begin{tcolorbox}[breakable, size=fbox, boxrule=1pt, pad at break*=1mm,colback=cellbackground, colframe=cellborder]
\prompt{In}{incolor}{10}{\boxspacing}
\begin{Verbatim}[commandchars=\\\{\}]
\PY{n}{mfi}\PY{o}{.}\PY{n}{plotIm}\PY{p}{(}\PY{p}{)}
\end{Verbatim}
\end{tcolorbox}

    \begin{center}
    \adjustimage{max size={0.9\linewidth}{0.9\paperheight}}{socfin_m_files/socfin_m_45_0.png}
    \end{center}
    { \hspace*{\fill} \\}
    
    \begin{tcolorbox}[breakable, size=fbox, boxrule=1pt, pad at break*=1mm,colback=cellbackground, colframe=cellborder]
\prompt{In}{incolor}{11}{\boxspacing}
\begin{Verbatim}[commandchars=\\\{\}]
\PY{n}{mfi}\PY{o}{.}\PY{n}{AM}\PY{p}{(}\PY{l+m+mi}{0}\PY{p}{)}\PY{p}{,} \PY{n}{mfi}\PY{o}{.}\PY{n}{AMe}\PY{p}{(}\PY{l+m+mi}{0}\PY{p}{)}
\end{Verbatim}
\end{tcolorbox}

            \begin{tcolorbox}[breakable, size=fbox, boxrule=.5pt, pad at break*=1mm, opacityfill=0]
\prompt{Out}{outcolor}{11}{\boxspacing}
\begin{Verbatim}[commandchars=\\\{\}]
(129.99999999999997, 149.99999999999997)
\end{Verbatim}
\end{tcolorbox}
        
    \begin{tcolorbox}[breakable, size=fbox, boxrule=1pt, pad at break*=1mm,colback=cellbackground, colframe=cellborder]
\prompt{In}{incolor}{12}{\boxspacing}
\begin{Verbatim}[commandchars=\\\{\}]
\PY{n}{plt}\PY{o}{.}\PY{n}{figure}\PY{p}{(}\PY{n}{figsize}\PY{o}{=}\PY{p}{(}\PY{l+m+mi}{4}\PY{p}{,} \PY{l+m+mi}{3}\PY{p}{)}\PY{p}{)}
\PY{n}{mfi}\PY{o}{.}\PY{n}{plotDE}\PY{p}{(}\PY{n}{mfi}\PY{o}{.}\PY{n}{beta}\PY{p}{)}
\end{Verbatim}
\end{tcolorbox}

    \begin{Verbatim}[commandchars=\\\{\}]
h:\textbackslash{}My Drive\textbackslash{}jpapers\textbackslash{}siv\textbackslash{}social-finance\textbackslash{}notebooks\textbackslash{}socialfinance.py:202:
RuntimeWarning: divide by zero encountered in divide
  de = (I + F - Im) / Im
    \end{Verbatim}

    \begin{center}
    \adjustimage{max size={0.9\linewidth}{0.9\paperheight}}{socfin_m_files/socfin_m_47_1.png}
    \end{center}
    { \hspace*{\fill} \\}
    
    \subsection{The borrower/entrepreneur's expected
return}\label{the-borrowerentrepreneurs-expected-return}

Is what is left after all others have been paid.

    If it is an equity only loan then the expected return is:

\[E(s|p) = pX - \beta \cdot I  - \beta F  - m^e(A)\]

    If it's a leveraged loan then the expected return is:

    \[E(s|p) = pX - \gamma \cdot I  - \gamma F  - m(A) \left ( 
  1 + \frac{\beta-\gamma}{\beta} \frac{q}{p-q} 
 \right )  \]

    \begin{tcolorbox}[breakable, size=fbox, boxrule=1pt, pad at break*=1mm,colback=cellbackground, colframe=cellborder]
\prompt{In}{incolor}{13}{\boxspacing}
\begin{Verbatim}[commandchars=\\\{\}]
\PY{n}{beta}\PY{p}{,} \PY{n}{gamma}\PY{p}{,} \PY{n}{p}\PY{p}{,} \PY{n}{q} \PY{o}{=} \PY{n}{mfi}\PY{o}{.}\PY{n}{beta}\PY{p}{,} \PY{n}{mfi}\PY{o}{.}\PY{n}{gamma}\PY{p}{,} \PY{n}{mfi}\PY{o}{.}\PY{n}{p}\PY{p}{,} \PY{n}{mfi}\PY{o}{.}\PY{n}{q}   
\PY{p}{(}\PY{l+m+mi}{1}\PY{o}{+} \PY{p}{(}\PY{n}{beta} \PY{o}{\PYZhy{}} \PY{n}{gamma}\PY{p}{)} \PY{o}{*} \PY{n}{q} \PY{o}{/} \PY{p}{(}\PY{n}{beta} \PY{o}{*} \PY{p}{(}\PY{n}{p} \PY{o}{\PYZhy{}} \PY{n}{q}\PY{p}{)}\PY{p}{)}\PY{p}{)} 
\end{Verbatim}
\end{tcolorbox}

            \begin{tcolorbox}[breakable, size=fbox, boxrule=.5pt, pad at break*=1mm, opacityfill=0]
\prompt{Out}{outcolor}{13}{\boxspacing}
\begin{Verbatim}[commandchars=\\\{\}]
1.9111111111111108
\end{Verbatim}
\end{tcolorbox}
        
    \subsection{The effect of subsidizing fixed
costs}\label{the-effect-of-subsidizing-fixed-costs}

Here we study the effects of a subsidy to fixed costs. We can predict
the following four impacts or transformations. Where G

    \begin{longtable}[]{@{}
  >{\raggedright\arraybackslash}p{(\columnwidth - 4\tabcolsep) * \real{0.3529}}
  >{\raggedright\arraybackslash}p{(\columnwidth - 4\tabcolsep) * \real{0.3529}}
  >{\raggedright\arraybackslash}p{(\columnwidth - 4\tabcolsep) * \real{0.2941}}@{}}
\toprule\noalign{}
\begin{minipage}[b]{\linewidth}\raggedright
Impact Group
\end{minipage} & \begin{minipage}[b]{\linewidth}\raggedright
Definition
\end{minipage} & \begin{minipage}[b]{\linewidth}\raggedright
notes
\end{minipage} \\
\midrule\noalign{}
\endhead
\bottomrule\noalign{}
\endlastfoot
A & Group 1 --\textgreater{} Group 2 & new MFIs where none before \\
B & Group 2 --\textgreater{} Group 2 & support existing MFIs , no new
leverage \\
C & Group 2 --\textgreater{} Group 3 & transform equity-only to
leveraged MFI \\
D & Group 3 --\textgreater{} Group 3 or 4 & increase leverage at already
leveraged \\
\end{longtable}

    where the original groups are:

\begin{longtable}[]{@{}ll@{}}
\toprule\noalign{}
Group & Definition \\
\midrule\noalign{}
\endhead
\bottomrule\noalign{}
\endlastfoot
1 & no lending \\
2 & equity-only MFI \\
3 & leveraged MFI \\
4 & direct banking \\
\end{longtable}

Let us look at the effect of a fixed cost subsidy (reduce F = 30 to F =
10) on borrower returns.

\subsubsection{Subsidy impact on return to
borrower:}\label{subsidy-impact-on-return-to-borrower}

    Let's first look at the borrower return by pledgeable assets level
\(A\). Borrowers with more pledgeable assets capture more of the project
returns \(EX -\gamma I - F\). That's because part of the surplus goes to
monitoring (or rents) for lower \(A\) borrowers.

    \begin{tcolorbox}[breakable, size=fbox, boxrule=1pt, pad at break*=1mm,colback=cellbackground, colframe=cellborder]
\prompt{In}{incolor}{14}{\boxspacing}
\begin{Verbatim}[commandchars=\\\{\}]
\PY{n}{br1} \PY{o}{=} \PY{n}{mfi}\PY{o}{.}\PY{n}{breturn}\PY{p}{(}\PY{n}{A}\PY{p}{)} 
\PY{n}{plt}\PY{o}{.}\PY{n}{figure}\PY{p}{(}\PY{n}{figsize}\PY{o}{=}\PY{p}{(}\PY{l+m+mi}{4}\PY{p}{,} \PY{l+m+mi}{3}\PY{p}{)}\PY{p}{)}
\PY{n}{plt}\PY{o}{.}\PY{n}{plot}\PY{p}{(}\PY{n}{A}\PY{p}{,}\PY{n}{br1}\PY{p}{)}
\PY{n}{plt}\PY{o}{.}\PY{n}{xlabel}\PY{p}{(}\PY{l+s+s1}{\PYZsq{}}\PY{l+s+s1}{A}\PY{l+s+s1}{\PYZsq{}}\PY{p}{)}
\PY{n}{plt}\PY{o}{.}\PY{n}{grid}\PY{p}{(}\PY{p}{)}
\PY{n}{plt}\PY{o}{.}\PY{n}{ylabel}\PY{p}{(}\PY{l+s+s1}{\PYZsq{}}\PY{l+s+s1}{breturn}\PY{l+s+s1}{\PYZsq{}}\PY{p}{)}\PY{p}{;}
\end{Verbatim}
\end{tcolorbox}

    \begin{center}
    \adjustimage{max size={0.9\linewidth}{0.9\paperheight}}{socfin_m_files/socfin_m_57_0.png}
    \end{center}
    { \hspace*{\fill} \\}
    
    \begin{tcolorbox}[breakable, size=fbox, boxrule=1pt, pad at break*=1mm,colback=cellbackground, colframe=cellborder]
\prompt{In}{incolor}{15}{\boxspacing}
\begin{Verbatim}[commandchars=\\\{\}]
\PY{n}{br1} \PY{o}{=} \PY{n}{mfi}\PY{o}{.}\PY{n}{breturn}\PY{p}{(}\PY{n}{A}\PY{p}{)}       \PY{c+c1}{\PYZsh{} borrower return in MFIs with F=20}
\PY{n}{mfi2} \PY{o}{=} \PY{n}{Bank}\PY{p}{(}\PY{n}{A}\PY{p}{,} \PY{n}{mfi}\PY{o}{.}\PY{n}{beta}\PY{p}{)}             
\PY{n}{mfi2}\PY{o}{.}\PY{n}{f} \PY{o}{=} \PY{n}{mfi}\PY{o}{.}\PY{n}{f} \PY{o}{\PYZhy{}} \PY{l+m+mi}{10}
\PY{n}{br2} \PY{o}{=} \PY{n}{mfi2}\PY{o}{.}\PY{n}{breturn}\PY{p}{(}\PY{n}{A}\PY{p}{)}     \PY{c+c1}{\PYZsh{} borrower return in MFIs with F=10a}
\end{Verbatim}
\end{tcolorbox}

    \begin{tcolorbox}[breakable, size=fbox, boxrule=1pt, pad at break*=1mm,colback=cellbackground, colframe=cellborder]
\prompt{In}{incolor}{16}{\boxspacing}
\begin{Verbatim}[commandchars=\\\{\}]
\PY{n}{p}\PY{p}{,} \PY{n}{q}\PY{p}{,} \PY{n}{gam}\PY{p}{,} \PY{n}{beta} \PY{o}{=} \PY{n}{mfi}\PY{o}{.}\PY{n}{p}\PY{p}{,} \PY{n}{mfi}\PY{o}{.}\PY{n}{q}\PY{p}{,} \PY{n}{mfi}\PY{o}{.}\PY{n}{gamma}\PY{p}{,} \PY{n}{mfi}\PY{o}{.}\PY{n}{beta}
\PY{n}{plt}\PY{o}{.}\PY{n}{plot}\PY{p}{(}\PY{n}{A}\PY{p}{,} \PY{n}{mfi}\PY{o}{.}\PY{n}{mon}\PY{p}{(}\PY{n}{A}\PY{p}{)}\PY{o}{*}\PY{p}{(}\PY{l+m+mi}{1} \PY{o}{+} \PY{p}{(}\PY{p}{(}\PY{n}{beta} \PY{o}{\PYZhy{}} \PY{n}{gam}\PY{p}{)} \PY{o}{/} \PY{n}{beta}\PY{p}{)} \PY{o}{*} \PY{p}{(}\PY{n}{q} \PY{o}{/} \PY{p}{(}\PY{n}{p} \PY{o}{\PYZhy{}} \PY{n}{q}\PY{p}{)}\PY{p}{)}\PY{p}{)}\PY{p}{)}
\PY{n}{plt}\PY{o}{.}\PY{n}{plot}\PY{p}{(}\PY{n}{A}\PY{p}{,} \PY{n}{mfi2}\PY{o}{.}\PY{n}{mon}\PY{p}{(}\PY{n}{A}\PY{p}{)}\PY{o}{*}\PY{p}{(}\PY{l+m+mi}{1} \PY{o}{+} \PY{p}{(}\PY{p}{(}\PY{n}{beta} \PY{o}{\PYZhy{}} \PY{n}{gam}\PY{p}{)} \PY{o}{/} \PY{n}{beta}\PY{p}{)} \PY{o}{*} \PY{p}{(}\PY{n}{q} \PY{o}{/} \PY{p}{(}\PY{n}{p} \PY{o}{\PYZhy{}} \PY{n}{q}\PY{p}{)}\PY{p}{)}\PY{p}{)}\PY{p}{)}
\PY{c+c1}{\PYZsh{}plt.plot(A,mfi.monE(A));}
\end{Verbatim}
\end{tcolorbox}

            \begin{tcolorbox}[breakable, size=fbox, boxrule=.5pt, pad at break*=1mm, opacityfill=0]
\prompt{Out}{outcolor}{16}{\boxspacing}
\begin{Verbatim}[commandchars=\\\{\}]
[<matplotlib.lines.Line2D at 0x1a744f24490>]
\end{Verbatim}
\end{tcolorbox}
        
    \begin{center}
    \adjustimage{max size={0.9\linewidth}{0.9\paperheight}}{socfin_m_files/socfin_m_59_1.png}
    \end{center}
    { \hspace*{\fill} \\}
    
    \begin{tcolorbox}[breakable, size=fbox, boxrule=1pt, pad at break*=1mm,colback=cellbackground, colframe=cellborder]
\prompt{In}{incolor}{17}{\boxspacing}
\begin{Verbatim}[commandchars=\\\{\}]
\PY{n}{plt}\PY{o}{.}\PY{n}{plot}\PY{p}{(}\PY{n}{A}\PY{p}{,} \PY{n}{br1}\PY{p}{)}
\PY{n}{plt}\PY{o}{.}\PY{n}{plot}\PY{p}{(}\PY{n}{A}\PY{p}{,} \PY{n}{br2}\PY{p}{)}
\PY{n}{plt}\PY{o}{.}\PY{n}{plot}\PY{p}{(}\PY{n}{A}\PY{p}{,} \PY{n}{br2}\PY{o}{\PYZhy{}}\PY{n}{br1}\PY{p}{)}                   \PY{c+c1}{\PYZsh{} change in borrower return}
\PY{n}{plt}\PY{o}{.}\PY{n}{title}\PY{p}{(}\PY{l+s+s1}{\PYZsq{}}\PY{l+s+s1}{Borrower returns at F=30 and F=20}\PY{l+s+s1}{\PYZsq{}}\PY{p}{)}
\PY{n}{plt}\PY{o}{.}\PY{n}{xlabel}\PY{p}{(}\PY{l+s+s1}{\PYZsq{}}\PY{l+s+s1}{A \PYZhy{}\PYZhy{} pledgeable assets}\PY{l+s+s1}{\PYZsq{}}\PY{p}{)}
\PY{n}{plt}\PY{o}{.}\PY{n}{axvline}\PY{p}{(}\PY{n}{x}\PY{o}{=}\PY{n}{mfi}\PY{o}{.}\PY{n}{Amin}\PY{p}{(}\PY{p}{)}\PY{p}{,} \PY{n}{linestyle} \PY{o}{=}\PY{l+s+s1}{\PYZsq{}}\PY{l+s+s1}{:}\PY{l+s+s1}{\PYZsq{}}\PY{p}{)}
\PY{n}{plt}\PY{o}{.}\PY{n}{axvline}\PY{p}{(}\PY{n}{x}\PY{o}{=}\PY{n}{mfi}\PY{o}{.}\PY{n}{Across}\PY{p}{(}\PY{p}{)}\PY{p}{,} \PY{n}{linestyle} \PY{o}{=}\PY{l+s+s1}{\PYZsq{}}\PY{l+s+s1}{:}\PY{l+s+s1}{\PYZsq{}}\PY{p}{)}
\PY{n}{plt}\PY{o}{.}\PY{n}{axvline}\PY{p}{(}\PY{n}{x}\PY{o}{=}\PY{n}{mfi}\PY{o}{.}\PY{n}{AM}\PY{p}{(}\PY{l+m+mi}{0}\PY{p}{)}\PY{p}{,} \PY{n}{linestyle} \PY{o}{=}\PY{l+s+s1}{\PYZsq{}}\PY{l+s+s1}{:}\PY{l+s+s1}{\PYZsq{}}\PY{p}{)}
\PY{n}{plt}\PY{o}{.}\PY{n}{axhline}\PY{p}{(}\PY{n}{y}\PY{o}{=}\PY{n}{mfi}\PY{o}{.}\PY{n}{f}\PY{o}{\PYZhy{}}\PY{n}{mfi2}\PY{o}{.}\PY{n}{f}\PY{p}{,}\PY{n}{linestyle}\PY{o}{=}\PY{l+s+s1}{\PYZsq{}}\PY{l+s+s1}{:}\PY{l+s+s1}{\PYZsq{}}\PY{p}{)}\PY{p}{;}     \PY{c+c1}{\PYZsh{} dashed line at subsidy level (20=20\PYZhy{}10)}
\PY{n}{plt}\PY{o}{.}\PY{n}{xlim}\PY{p}{(}\PY{l+m+mi}{0}\PY{p}{,}\PY{n+nb}{max}\PY{p}{(}\PY{n}{A}\PY{p}{)}\PY{p}{)}\PY{p}{,} \PY{n}{plt}\PY{o}{.}\PY{n}{ylim}\PY{p}{(}\PY{l+m+mi}{0}\PY{p}{,} \PY{l+m+mi}{120}\PY{p}{)}
\PY{n}{plt}\PY{o}{.}\PY{n}{grid}\PY{p}{(}\PY{n}{axis}\PY{o}{=}\PY{l+s+s1}{\PYZsq{}}\PY{l+s+s1}{y}\PY{l+s+s1}{\PYZsq{}}\PY{p}{)}
\end{Verbatim}
\end{tcolorbox}

    \begin{center}
    \adjustimage{max size={0.9\linewidth}{0.9\paperheight}}{socfin_m_files/socfin_m_60_0.png}
    \end{center}
    { \hspace*{\fill} \\}
    
    \begin{tcolorbox}[breakable, size=fbox, boxrule=1pt, pad at break*=1mm,colback=cellbackground, colframe=cellborder]
\prompt{In}{incolor}{18}{\boxspacing}
\begin{Verbatim}[commandchars=\\\{\}]
\PY{n}{plt}\PY{o}{.}\PY{n}{plot}\PY{p}{(}\PY{n}{A}\PY{p}{,} \PY{n}{br1}\PY{p}{)}
\PY{n}{plt}\PY{o}{.}\PY{n}{plot}\PY{p}{(}\PY{n}{A}\PY{p}{,} \PY{n}{br2}\PY{p}{)}
\PY{n}{plt}\PY{o}{.}\PY{n}{plot}\PY{p}{(}\PY{n}{A}\PY{p}{,} \PY{n}{br2}\PY{o}{\PYZhy{}}\PY{n}{br1}\PY{p}{)}                   \PY{c+c1}{\PYZsh{} change in borrower return}
\PY{n}{plt}\PY{o}{.}\PY{n}{title}\PY{p}{(}\PY{l+s+s1}{\PYZsq{}}\PY{l+s+s1}{Borrower returns at F=20, F=10 and difference}\PY{l+s+s1}{\PYZsq{}}\PY{p}{)}
\PY{n}{plt}\PY{o}{.}\PY{n}{xlabel}\PY{p}{(}\PY{l+s+s1}{\PYZsq{}}\PY{l+s+s1}{A \PYZhy{}\PYZhy{} pledgeable assets}\PY{l+s+s1}{\PYZsq{}}\PY{p}{)}
\PY{n}{plt}\PY{o}{.}\PY{n}{axvline}\PY{p}{(}\PY{n}{x}\PY{o}{=}\PY{n}{mfi}\PY{o}{.}\PY{n}{Amin}\PY{p}{(}\PY{p}{)}\PY{p}{,} \PY{n}{linestyle} \PY{o}{=}\PY{l+s+s1}{\PYZsq{}}\PY{l+s+s1}{:}\PY{l+s+s1}{\PYZsq{}}\PY{p}{)}
\PY{n}{plt}\PY{o}{.}\PY{n}{axvline}\PY{p}{(}\PY{n}{x}\PY{o}{=}\PY{n}{mfi}\PY{o}{.}\PY{n}{Across}\PY{p}{(}\PY{p}{)}\PY{p}{,} \PY{n}{linestyle} \PY{o}{=}\PY{l+s+s1}{\PYZsq{}}\PY{l+s+s1}{:}\PY{l+s+s1}{\PYZsq{}}\PY{p}{)}
\PY{n}{plt}\PY{o}{.}\PY{n}{axvline}\PY{p}{(}\PY{n}{x}\PY{o}{=}\PY{n}{mfi}\PY{o}{.}\PY{n}{AM}\PY{p}{(}\PY{l+m+mi}{0}\PY{p}{)}\PY{p}{,} \PY{n}{linestyle} \PY{o}{=}\PY{l+s+s1}{\PYZsq{}}\PY{l+s+s1}{:}\PY{l+s+s1}{\PYZsq{}}\PY{p}{)}
\PY{n}{plt}\PY{o}{.}\PY{n}{axhline}\PY{p}{(}\PY{n}{y}\PY{o}{=}\PY{n}{mfi}\PY{o}{.}\PY{n}{F}\PY{o}{\PYZhy{}}\PY{n}{mfi2}\PY{o}{.}\PY{n}{F}\PY{p}{,}\PY{n}{linestyle}\PY{o}{=}\PY{l+s+s1}{\PYZsq{}}\PY{l+s+s1}{:}\PY{l+s+s1}{\PYZsq{}}\PY{p}{)}\PY{p}{;}     \PY{c+c1}{\PYZsh{} dashed line at subsidy level (20=20\PYZhy{}10)}
\PY{n}{plt}\PY{o}{.}\PY{n}{xlim}\PY{p}{(}\PY{l+m+mi}{0}\PY{p}{,} \PY{n+nb}{max}\PY{p}{(}\PY{n}{A}\PY{p}{)}\PY{p}{)}\PY{p}{,} \PY{n}{plt}\PY{o}{.}\PY{n}{ylim}\PY{p}{(}\PY{l+m+mi}{0}\PY{p}{,} \PY{l+m+mi}{90}\PY{p}{)}
\PY{n}{plt}\PY{o}{.}\PY{n}{grid}\PY{p}{(}\PY{p}{)}
\end{Verbatim}
\end{tcolorbox}

    \begin{center}
    \adjustimage{max size={0.9\linewidth}{0.9\paperheight}}{socfin_m_files/socfin_m_61_0.png}
    \end{center}
    { \hspace*{\fill} \\}
    
    \textbf{Analysis:}

\begin{longtable}[]{@{}
  >{\raggedright\arraybackslash}p{(\columnwidth - 4\tabcolsep) * \real{0.3529}}
  >{\raggedright\arraybackslash}p{(\columnwidth - 4\tabcolsep) * \real{0.3529}}
  >{\raggedright\arraybackslash}p{(\columnwidth - 4\tabcolsep) * \real{0.2941}}@{}}
\toprule\noalign{}
\begin{minipage}[b]{\linewidth}\raggedright
Impact Group
\end{minipage} & \begin{minipage}[b]{\linewidth}\raggedright
description
\end{minipage} & \begin{minipage}[b]{\linewidth}\raggedright
notes
\end{minipage} \\
\midrule\noalign{}
\endhead
\bottomrule\noalign{}
\endlastfoot
A & previously excluded borrowers & only those with higher A get
benefits \textgreater{} subsidy \\
B & existing no-leverage borrowers & benefit\textgreater subsidy \\
C & borrowers in transformed MFI & leverage gives extra kick to subsidy,
increasing with A \\
D & leveraged to direct borrowing & impact greater than subsidy but
declining with A \\
E & Group 4 --\textgreater{} Group 4 & subsidy passed through 1 for 1 \\
\end{longtable}

    \subsubsection{Subsidy impact on number of
borrowers:}\label{subsidy-impact-on-number-of-borrowers}

We are looking at fixed-cost per loan F.

\[ N = \frac{K}{I^m +F} \]

    \begin{tcolorbox}[breakable, size=fbox, boxrule=1pt, pad at break*=1mm,colback=cellbackground, colframe=cellborder]
\prompt{In}{incolor}{19}{\boxspacing}
\begin{Verbatim}[commandchars=\\\{\}]
\PY{n}{nr1} \PY{o}{=} \PY{n}{mfi}\PY{o}{.}\PY{n}{nreach}\PY{p}{(}\PY{n}{A}\PY{p}{)}       \PY{c+c1}{\PYZsh{}number borrowers in MFIs with F=20}
\PY{n}{nr2} \PY{o}{=} \PY{n}{mfi2}\PY{o}{.}\PY{n}{nreach}\PY{p}{(}\PY{n}{A}\PY{p}{)}      \PY{c+c1}{\PYZsh{}number borrowers in MFIs with F=10}

\PY{c+c1}{\PYZsh{}plt.plot(A,nr1)}
\PY{c+c1}{\PYZsh{}plt.plot(A,nr2)}
\PY{n}{plt}\PY{o}{.}\PY{n}{plot}\PY{p}{(}\PY{n}{A}\PY{p}{,} \PY{n}{nr2}\PY{o}{\PYZhy{}}\PY{n}{nr1}\PY{p}{)}                   \PY{c+c1}{\PYZsh{} change in borrower return}
\PY{n}{plt}\PY{o}{.}\PY{n}{title}\PY{p}{(}\PY{l+s+s1}{\PYZsq{}}\PY{l+s+s1}{Change in number of borrowers when subsidize F=20 to F=10 and difference}\PY{l+s+s1}{\PYZsq{}}\PY{p}{)}
\PY{n}{plt}\PY{o}{.}\PY{n}{xlabel}\PY{p}{(}\PY{l+s+s1}{\PYZsq{}}\PY{l+s+s1}{A \PYZhy{}\PYZhy{} pledgeable assets}\PY{l+s+s1}{\PYZsq{}}\PY{p}{)}
\PY{n}{plt}\PY{o}{.}\PY{n}{axvline}\PY{p}{(}\PY{n}{x}\PY{o}{=}\PY{n}{mfi}\PY{o}{.}\PY{n}{Amin}\PY{p}{(}\PY{p}{)}\PY{p}{,} \PY{n}{linestyle} \PY{o}{=}\PY{l+s+s1}{\PYZsq{}}\PY{l+s+s1}{:}\PY{l+s+s1}{\PYZsq{}}\PY{p}{)}
\PY{n}{plt}\PY{o}{.}\PY{n}{axvline}\PY{p}{(}\PY{n}{x}\PY{o}{=}\PY{n}{mfi}\PY{o}{.}\PY{n}{Across}\PY{p}{(}\PY{p}{)}\PY{p}{,} \PY{n}{linestyle} \PY{o}{=}\PY{l+s+s1}{\PYZsq{}}\PY{l+s+s1}{:}\PY{l+s+s1}{\PYZsq{}}\PY{p}{)}
\PY{n}{plt}\PY{o}{.}\PY{n}{axvline}\PY{p}{(}\PY{n}{x}\PY{o}{=}\PY{n}{mfi}\PY{o}{.}\PY{n}{AM}\PY{p}{(}\PY{l+m+mi}{0}\PY{p}{)}\PY{p}{,} \PY{n}{linestyle} \PY{o}{=}\PY{l+s+s1}{\PYZsq{}}\PY{l+s+s1}{:}\PY{l+s+s1}{\PYZsq{}}\PY{p}{)}
\PY{n}{plt}\PY{o}{.}\PY{n}{axhline}\PY{p}{(}\PY{n}{y}\PY{o}{=}\PY{n}{mfi}\PY{o}{.}\PY{n}{F}\PY{o}{\PYZhy{}}\PY{n}{mfi2}\PY{o}{.}\PY{n}{F}\PY{p}{,}\PY{n}{linestyle}\PY{o}{=}\PY{l+s+s1}{\PYZsq{}}\PY{l+s+s1}{:}\PY{l+s+s1}{\PYZsq{}}\PY{p}{)}\PY{p}{;}     \PY{c+c1}{\PYZsh{} dashed line at subsidy level (20=20\PYZhy{}10)}
\end{Verbatim}
\end{tcolorbox}

    \begin{center}
    \adjustimage{max size={0.9\linewidth}{0.9\paperheight}}{socfin_m_files/socfin_m_64_0.png}
    \end{center}
    { \hspace*{\fill} \\}
    
    \begin{tcolorbox}[breakable, size=fbox, boxrule=1pt, pad at break*=1mm,colback=cellbackground, colframe=cellborder]
\prompt{In}{incolor}{20}{\boxspacing}
\begin{Verbatim}[commandchars=\\\{\}]
\PY{n}{nr1} \PY{o}{=} \PY{n}{mfi}\PY{o}{.}\PY{n}{nreach}\PY{p}{(}\PY{n}{A}\PY{p}{)}
\PY{n}{plt}\PY{o}{.}\PY{n}{plot}\PY{p}{(}\PY{n}{A}\PY{p}{,}\PY{n}{nr1}\PY{p}{)}
\PY{c+c1}{\PYZsh{}plt.plot(A,nr2)}
\PY{n}{plt}\PY{o}{.}\PY{n}{xlabel}\PY{p}{(}\PY{l+s+s1}{\PYZsq{}}\PY{l+s+s1}{A \PYZhy{}\PYZhy{} pledgeable assets}\PY{l+s+s1}{\PYZsq{}}\PY{p}{)}
\PY{n}{plt}\PY{o}{.}\PY{n}{xlim}\PY{p}{(}\PY{l+m+mi}{0}\PY{p}{,}\PY{l+m+mi}{110}\PY{p}{)}\PY{p}{,} \PY{n}{plt}\PY{o}{.}\PY{n}{ylim}\PY{p}{(}\PY{l+m+mi}{0}\PY{p}{,}\PY{l+m+mi}{400}\PY{p}{)}
\PY{n}{plt}\PY{o}{.}\PY{n}{axvline}\PY{p}{(}\PY{n}{x}\PY{o}{=}\PY{n}{mfi}\PY{o}{.}\PY{n}{Amin}\PY{p}{(}\PY{p}{)}\PY{p}{,} \PY{n}{linestyle} \PY{o}{=}\PY{l+s+s1}{\PYZsq{}}\PY{l+s+s1}{:}\PY{l+s+s1}{\PYZsq{}}\PY{p}{)}
\PY{n}{plt}\PY{o}{.}\PY{n}{axvline}\PY{p}{(}\PY{n}{x}\PY{o}{=}\PY{n}{mfi}\PY{o}{.}\PY{n}{Across}\PY{p}{(}\PY{p}{)}\PY{p}{,} \PY{n}{linestyle} \PY{o}{=}\PY{l+s+s1}{\PYZsq{}}\PY{l+s+s1}{:}\PY{l+s+s1}{\PYZsq{}}\PY{p}{)}
\PY{n}{plt}\PY{o}{.}\PY{n}{axvline}\PY{p}{(}\PY{n}{x}\PY{o}{=}\PY{n}{mfi}\PY{o}{.}\PY{n}{AM}\PY{p}{(}\PY{l+m+mi}{0}\PY{p}{)}\PY{p}{,} \PY{n}{linestyle} \PY{o}{=}\PY{l+s+s1}{\PYZsq{}}\PY{l+s+s1}{:}\PY{l+s+s1}{\PYZsq{}}\PY{p}{)}
\PY{n}{plt}\PY{o}{.}\PY{n}{axhline}\PY{p}{(}\PY{n}{y}\PY{o}{=}\PY{n}{mfi}\PY{o}{.}\PY{n}{F}\PY{o}{\PYZhy{}}\PY{n}{mfi2}\PY{o}{.}\PY{n}{F}\PY{p}{,}\PY{n}{linestyle}\PY{o}{=}\PY{l+s+s1}{\PYZsq{}}\PY{l+s+s1}{:}\PY{l+s+s1}{\PYZsq{}}\PY{p}{)}\PY{p}{;}     \PY{c+c1}{\PYZsh{} dashed line at subsidy level (20=20\PYZhy{}10)}
\end{Verbatim}
\end{tcolorbox}

    \begin{center}
    \adjustimage{max size={0.9\linewidth}{0.9\paperheight}}{socfin_m_files/socfin_m_65_0.png}
    \end{center}
    { \hspace*{\fill} \\}
    
    Analysis: Recall that K = 12000. So a non-leveraged lender can reach
\(N = 12000/(100+F)\). When \(F=20\) they can reach 100 borrowers. When
\(F=10\) they can reach 109 borrowers.

\textbf{Analysis:}

\begin{longtable}[]{@{}
  >{\raggedright\arraybackslash}p{(\columnwidth - 4\tabcolsep) * \real{0.3529}}
  >{\raggedright\arraybackslash}p{(\columnwidth - 4\tabcolsep) * \real{0.3529}}
  >{\raggedright\arraybackslash}p{(\columnwidth - 4\tabcolsep) * \real{0.2941}}@{}}
\toprule\noalign{}
\begin{minipage}[b]{\linewidth}\raggedright
Impact Group
\end{minipage} & \begin{minipage}[b]{\linewidth}\raggedright
description
\end{minipage} & \begin{minipage}[b]{\linewidth}\raggedright
notes
\end{minipage} \\
\midrule\noalign{}
\endhead
\bottomrule\noalign{}
\endlastfoot
A & previously excluded borrowers & 109 new borrowers in each of these
new MFIs \\
B & existing no-leverage borrowers & just 9 new borrowers per MFI \\
C, D & borrowers in transformed MFI & increase leverage gets many new,
increasing with A \\
\end{longtable}

note: need to explain/fix weird small dip at end.

    

    

    

    

    

    


    % Add a bibliography block to the postdoc
    
    
    
\end{document}
